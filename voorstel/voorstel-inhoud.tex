%---------- Inleiding ---------------------------------------------------------

% TODO: Is dit voorstel gebaseerd op een paper van Research Methods die je
% vorig jaar hebt ingediend? Heb je daarbij eventueel samengewerkt met een
% andere student?
% Zo ja, haal dan de tekst hieronder uit commentaar en pas aan.

%\paragraph{Opmerking}

% Dit voorstel is gebaseerd op het onderzoeksvoorstel dat werd geschreven in het
% kader van het vak Research Methods dat ik (vorig/dit) academiejaar heb
% uitgewerkt (met medesturent VOORNAAM NAAM als mede-auteur).
% 

\section{Inleiding}%
\label{sec:inleiding}
\newline
\colorbox{yellow}{(Probleemstelling)} Met de stijgende hoeveelheid online datacollectie en digitale diensten neemt ook de behoefte aan snelle en betrouwbare cybersecurity toe. Aangezien er veel meer cyberaanvallen zijn, dankzij o.a. AI attacks \autocite{Henderson2024}, worden de huidige systemen zoals Security Information and Event Management (SIEM) overspoeld met verschillende security-alerts zoals DNS, assets, logs, vulnerabilities, etc. 
Deze security-events worden vaak apart geanalyseerd, waardoor de verschillende verbanden tussen deze events verloren gaan. Om complexe aanvalspatronen te bestuderen en herkennen moeten security-events samen geanalyseerd worden om de detectie sneller en potentieel foutloos te maken. 
\newline
\newline
In deze paper worden Graph-based machine \newline learning technieken gebruikt om verbanden te zoeken tussen deze losse events, zodat cybersecurity bedrijven dit kunnen inschakelen in real-time dreigingsdetectie. Hiervoor wordt ook een proof-of-concept opgesteld met als doelstelling een snel en accuraat kettingdetectiesysteem op te stellen.

\subsection{Onderzoeksvragen en doelstellingen}
\label{sec:onderzoeksvragen}

Het doel van deze bachelorproef is om te onderzoeken hoe graph-based machine learning kan ingezet worden om security-events automatisch te linken en te correleren, zodat dreigingen sneller en nauwkeuriger kunnen worden herkend.  
Hieruit volgt de centrale onderzoeksvraag:
\newline
\newline
\colorbox{yellow}{
    \textbf{Hoofdonderzoeksvraag:}}
    Hoe kan graph-based machine learning worden toegepast om security-events automatisch te chainen en te correleren, zodat complexe aanvalspatronen sneller en accurater worden gedetecteerd?
\newline
\newline
Om deze hoofdvraag te beantwoorden, worden onderstaande deelvragen opgesteld.  
De deelvragen worden opgesplitst in twee domeinen: het \emph{probleemdomein}, dat focust op de context en uitdagingen van bestaande systemen, en het \emph{oplossingsdomein}, dat onderzoekt hoe graph-based technieken deze kunnen verbeteren.
\newline
\newline
    \textbf{\colorbox{yellow}{Deelvragen} met betrekking tot het probleemdomein:}
    \begin{enumerate}
        \item Welke soorten security-events en datavelden (zoals host, tijdstip, asset-ID of CVE) zijn relevant voor chaining?
        \item Hoe worden deze events momenteel gecorreleerd of geanalyseerd binnen Security Operations Centers (SOC’s), en wat zijn daarbij de beperkingen?
        \item Welke relaties en kenmerken zijn cruciaal om events semantisch met elkaar te verbinden (bijvoorbeeld tijd, asset of type dreiging)?
        \item Wat zijn de voor- en nadelen van bestaande methodes zoals rule-based correlatie of SIEM-regels?
\end{enumerate}
\newline
    \textbf{\colorbox{yellow}{Deelvragen} met betrekking tot het oplossingsdomein:}
    \begin{enumerate}
        \item Welke graph-based machine learning-\newline technieken (zoals Graph Neural Networks of graph clustering) zijn het meest geschikt om relaties tussen events te leren?
        \item Hoe kan een dataset van security-events worden voorbereid en omgezet in een grafenstructuur met nodes en edges?
        \item Hoe kan de performantie van het chaining-model worden geëvalueerd met objectieve metrics (zoals precision, recall etc.)?
        \item Hoe kan het prototype visueel worden gemaakt om aanvalspaden en verbanden tussen events duidelijk voor te stellen?
\end{enumerate}
\newline
\newline
Het beantwoorden van deze vragen moet leiden tot een werkend proof-of-concept dat aantoont hoe graph-based ML kan bijdragen aan snellere en contextueel rijkere dreigingsdetectie binnen bedrijfsomgevingen.



%---------- Stand van zaken ---------------------------------------------------

\section{Literatuurstudie}%
\label{sec:literatuurstudie}
\subsection{Traditionele Event Correlatie en SIEM}
Security Information and Event Management (SIEM) systemen worden vandaag het vaakst gebruikt om security-events te verzamelen, te ordenen en te analyseren. Ze halen data uit verschillende bronnen zoals logs, netwerkverkeer en applicaties \autocite{Vielberth2025}. SIEM’s gebruiken vaste regels (rule-based correlatie) die bijvoorbeeld tijd, host of type events combineren om mogelijke aanvallen te vinden. Deze aanpak werkt goed voor gekende patronen, maar volgens meerdere onderzoeken heeft ze beperkingen bij complexe en multiple layer attacks door het grote aantal events en de diversiteit van data \autocite{IBM2024}.  
\newline
\newline
Daarnaast zorgt het hoge aantal alerts dat SOC’s moeten opvolgen voor wat in de literatuur “alert fatigue” genoemd wordt: analisten krijgen te veel meldingen en missen daardoor soms echte dreigingen \autocite{Makhinova2025}.  
\newline
\newline
\newline
Daarom besluiten veel studies dat de klassieke systemen niet goed genoeg zijn voor moderne, contextuele dreigingsdetectie.
\newline
\newline
\subsection{AI en Machine Learning voor Threat Correlatie}
Om de beperkingen van rule-based systemen te verbeteren, gebruiken steeds meer organisaties machine learning (ML) en deep learning voor security-analyse \autocite{Gamage2025}.  
\newline
ML kan patronen ontdekken in grote datasets die mensen of vaste regels vaak niet zien. Daardoor kunnen aanvallen sneller en nauwkeuriger herkend worden. In recente reviews blijkt dat AI-technieken (zoals supervised ML en deep learning) vaak betere resultaten halen dan klassieke methodes, vooral bij complexe of grote hoeveelheden data \autocite{IBM2024}.  
\newline
\newline
Uit de literatuur blijkt ook dat bepaalde ML-
\newline
modellen, zoals convolutionele (CNN) en recurrente netwerken (RNN/LSTM), goed presteren bij het herkennen van opeenvolgende patronen of afwijkingen. Toch hebben ze moeite om relaties te vinden tussen verschillende entiteiten (zoals hosts of alerts) die structureel met elkaar verbonden zijn \autocite{Gamage2025}. Dat zorgt ervoor dat er steeds meer onderzoek komt naar graph-gebaseerde modellen.
\newline
\newline
\subsection{Graph-Based Machine Learning en Graph Neural Networks}
Graph-based machine learning (GBML)
\newline
en vooral Graph Neural Networks (GNNs) worden gezien als veelbelovende methodes om relaties tussen security-events te leren. In deze aanpak worden entiteiten zoals hosts, gebruikers of alerts voorgesteld als knooppunten (nodes), en hun relaties als verbindingen (edges) in een graaf. Op die manier kan het model verbanden leren die in traditionele ML-modellen moeilijk te zien zijn \autocite{Kanka2022}.  
\newline
\newline
Recente onderzoeken tonen aan dat GNN’s beter kunnen omgaan met complexere correlaties die traditionele systemen vaak missen. Zo beschrijft Brooks et al. een GNN-model dat relaties tussen logbronnen en gedragingen leert en daarmee gecoördineerde cyberaanvallen kan herkennen met hogere precisie en betrouwbaarheid dan klassieke correlatie-engines \autocite{Brooks2025}.  
Andere studies gebruiken graph-matching technieken om alerts te groeperen in graph clusters, wat helpt om meldingen beter te begrijpen en te prioriteren \autocite{Vitulyova2025}.
\newline
\newline
\subsection{Hybride en Geavanceerde Benaderingen}
Sommige onderzoeken bouwen verder op \newline graph-modellen door ze te combineren met andere ML-technieken. Vitulyova et al. tonen bijvoorbeeld een hybride model dat GNN’s combineert met Long Short-Term Memory (LSTM) netwerken. Zo kunnen ze zowel de structuur (relaties tussen entiteiten) als de tijdsvolgorde van een aanval meenemen \autocite{Vitulyova2025}. Hun resultaten tonen dat zulke modellen erg goed kunnen voorspellen hoe een aanval zich ontwikkelt.  
\newline
\newline
Een recente literatuurstudie laat zien dat er vandaag veel soorten correlatiemethoden bestaan van rule-based tot statistisch en AI-gebaseerd maar dat er nog steeds een tekort is aan oplossingen die schaalbaar én contextueel zijn voor real-time detectie \autocite{Kanka2022}. Dit ondersteunt het idee dat graph-gebaseerde en hybride AI- \newline aanpakken een belangrijk onderwerp blijven voor toekomstig onderzoek.
\newline
\newline
\subsection{Knelpunten en Onderzoeksopeningen}
Een vaak vermeld probleem in de literatuur is het evenwicht tussen prestatie en uitlegbaarheid. Sommige deep learning-modellen halen sterke resultaten, maar hun werking is moeilijk te \newline begrijpen. Dat is een nadeel in SOC-omgevingen waar beslissingen moeten kunnen worden verantwoord \autocite{Gamage2025}.  
Daarnaast is er nog de uitdaging van schaalbaarheid, want moderne bedrijven genereren enorme hoeveelheden data die snel en efficiënt moeten worden verwerkt.
\newline
\newline
Samengevat tonen de onderzochte studies dat:
\begin{itemize}
    \item Traditionele SIEM- en rule-based systemen nuttig zijn, maar tekortschieten bij complexe aanvalspatronen;
    \item AI en machine learning veel mogelijkheden bieden om event correlatie te verbeteren;
    \item Graph-based machine learning, en vooral \newline GNN’s, steeds vaker onderzocht worden omdat ze relaties in data beter kunnen begrijpen;
    \item Hybride modellen het potentieel hebben om zowel structuur als tijdsaspecten te leren;
    \item Er nog open vragen zijn rond uitlegbaarheid, real-time schaalbaarheid en praktische integratie.
\end{itemize}


%---------- Methodologie ------------------------------------------------------
\section{Methodologie}
\label{sec:methodologie}

Het doel van dit onderzoek is om te onderzoeken hoe graph-based machine learning (GBML) kan gebruikt worden om security-events automatisch te linken of te “chainen”.  
De bachelorproef heeft een toegepast en technisch karakter, waarbij een proof-of-concept (PoC) wordt ontwikkeld en getest op realistische data.  
De onderzoeksaanpak bestaat uit drie fasen: een analyse van het probleem en de vereisten, de ontwikkeling van het prototype, en een evaluatie van de resultaten.  
\newline
\newline
Elke fase levert een concreet resultaat of \emph{deliverable} op. Deze aanpak sluit aan bij eerder onderzoek dat graph-based technieken inzet voor correlatie van security-data \autocite{Kanka2022, Brooks2025, Vitulyova2025}.
\newline
\newline

\subsection{Fase 1: Analyse van het probleemdomein}
In de eerste fase wordt de probleemcontext verder onderzocht.  
Er wordt bepaald welke soorten security-events relevant zijn voor chaining, zoals logs, DNS-data, asset-informatie, kwetsbaarheden (CVE’s) en alerts uit SIEM-systemen.  
Daarnaast wordt onderzocht welke datavelden nodig zijn (bijv. tijdstip, host, IP-adres, asset-ID, type dreiging) om verbanden tussen events te herkennen.  
\newline
\newline
Er wordt een kleine literatuurstudie uitgevoerd om te bekijken hoe events momenteel gecorreleerd worden in Security Operations Centers (SOC’s) en welke beperkingen er bestaan bij rule-based systemen.  
Ook worden bestaande frameworks zoals MITRE ATT\&CK en NIST SP 800-215 geraadpleegd, die beschrijven hoe aanvalspatronen en threat intelligence kunnen worden gekoppeld \autocite{MITRE2024, NIST2023}.  
\newline
\newline
Deze fase resulteert in een \emph{requirements-document} waarin staat welke data en relaties het model moet ondersteunen en welke meetbare doelen (zoals detectiesnelheid of precisie) zullen gebruikt worden tijdens de evaluatie.

\subsection{Fase 2: Ontwikkeling van het proof-of-concept}
De tweede fase bestaat uit de praktische uitwerking van het prototype.  
Er wordt een synthetische dataset opgebouwd die echte security-events nabootst, zonder bedrijfsgevoelige data te gebruiken.  
Elke gebeurtenis wordt voorgesteld als een node, en relaties (zoals gedeeld IP, tijdsvenster of asset) worden als edges toegevoegd.  
Deze dataset vormt de basis voor de grafen waarop het chaining-model zal leren.
\newline
\newline
Vervolgens wordt een graph-based machine learning model ontwikkeld, zoals een Graph Neural Network (GNN) of graph-clustering algoritme.  
De implementatie gebeurt in \textbf{Python}, met behulp van libraries zoals:
\begin{itemize}
    \item \textbf{PyTorch Geometric}: voor graph neural networks en embeddings;
    \item \textbf{NetworkX}: voor het opbouwen en analyseren van graafstructuren;
    \item \textbf{Pandas / NumPy}: voor data cleaning en preprocessing;
    \item \textbf{Matplotlib / Plotly}: voor visualisatie van \newline chaining-resultaten.
\end{itemize}
\newline
Daarnaast wordt gekeken naar open-source dreigingsdata (zoals MISP of MITRE ATT\&CK-samples) om realistische eventtypes te simuleren.  
Het prototype zal de chaining visualiseren als een interactief netwerk waarin nodes (events) met elkaar verbonden worden.  
\newline
\newline
De output van deze fase is een eerste werkende PoC die automatisch verbanden legt tussen events op basis van hun eigenschappen en relaties.

\subsection{Fase 3: Evaluatie en validatie}
In de derde fase wordt de performantie van het model geëvalueerd met objectieve metrics zoals:
\begin{itemize}
    \item \textbf{Precision en recall}: om de juistheid van detectie te meten;
    \item \textbf{F1-score}: om de balans tussen precisie en volledigheid te evalueren;
    \item \textbf{Tijd-tot-detectie}: om de snelheid van chaining te meten.
\end{itemize}
\newline
De resultaten worden vergeleken met een eenvoudige baseline (bijv. rule-based correlatie of random linking) om te zien of GBML effectief betere resultaten oplevert.  
Er wordt ook gekeken naar uitlegbaarheid: kan het model verduidelijken waarom twee events aan elkaar gelinkt worden, of blijft dit een “black box”?  
\newline
\newline
De output van deze fase is een \emph{evaluatierapport} met meetbare resultaten, grafieken en een visuele voorstelling van de chaining-output in de vorm van een netwerkdiagram.

\subsection{Planning en tijdsinschatting}
Het volledige onderzoek loopt over ongeveer 14 weken.  
De globale planning ziet er als volgt uit:
\begin{itemize}
    \item \textbf{Week 1–3:} Analyse van het probleem, literatuuronderzoek en data-vereisten opstellen.
    \item \textbf{Week 4–7:} Ontwikkeling van de dataset en het proof-of-concept in Python.
    \item \textbf{Week 8–11:} Testen, finetunen en documenteren van het model.
    \item \textbf{Week 12–14:} Evaluatie, resultaten visualiseren en rapporteren.
\end{itemize}
Elke fase eindigt met een concreet deliverable: een requirements-document, een werkend prototype en een evaluatierapport met resultaten en conclusies.

\subsection{Onderzoekstype}
Deze bachelorproef valt onder de categorie \textbf{toegepast en experimenteel onderzoek}.  
Het onderzoek combineert literatuurstudie en praktische implementatie in de vorm van een proof-of-concept.  
De aanpak is meetbaar en reproduceerbaar, met focus op technische uitvoering, waardoor het voldoende diepgang biedt voor een bachelor toegepaste informatica.
\newline
\newline
\newline
\newline
\newline
\newline

%---------- Verwachte resultaten ----------------------------------------------
\section{Conclusie en verwachte resultaten}
\label{sec:conclusie}

Uit dit onderzoek wordt verwacht dat graph-based machine learning (GBML) een duidelijke meerwaarde kan bieden voor het automatisch linken van security-events.  
Door gebruik te maken van grafenstructuren en technieken zoals Graph Neural Networks (GNN), kan het systeem verbanden leren die niet rechtstreeks zichtbaar zijn in traditionele SIEM-oplossingen of rule-based correlatie.  
\newline
\newline
Het proof-of-concept zal waarschijnlijk aantonen dat graph-based modellen een hogere precisie en betere contextherkenning opleveren bij het detecteren van complexe aanvalspatronen, vergeleken met bestaande methodes.  
Volgens eerdere studies \autocite{Vitulyova2025, Gamage2025} kunnen zulke modellen subtiele relaties tussen events ontdekken, zoals overeenkomsten in tijd, asset of gedrag, wat leidt tot minder false positives en snellere detectie.
\newline
\newline
De verwachte resultaten kunnen schematisch worden weergegeven door een vergelijking van klassieke en graph-based detectie:
\begin{itemize}
    \item Op de x-as: de hoeveelheid verwerkte events of complexiteit van de dataset;
    \item Op de y-as: de nauwkeurigheid of F1-score van de detectie.
\end{itemize}
Er wordt verwacht dat de curve van het GBML-model trager stijgt in verwerkingskost, maar sneller hogere detectiekwaliteit bereikt dan het rule-based model.  
De resultaten zullen gevisualiseerd worden in grafieken die de prestatieverschillen (precision, recall, F1-score) tonen tussen beide methodes.
\newline
\newline
De meerwaarde voor de doelgroep, met name cybersecuritybedrijven zoals Secutec en SOC-\newline analisten, ligt in het feit dat deze aanpak toelaat om sneller verbanden te zien tussen schijnbare losse alerts.  
Dat kan leiden tot kortere reactietijden, een daling van onnodige meldingen, en een beter overzicht van dreigingen binnen een organisatie.  
\newline
\newline
Hoewel er nog uitdagingen verwacht worden op vlak van schaalbaarheid en uitlegbaarheid, biedt het onderzoek een praktisch inzicht in hoe graph-based ML in de toekomst geïntegreerd kan worden in bestaande detectiesystemen.  
De proof-of-concept zal dus niet alleen technische haalbaarheid aantonen, maar ook dienen als eerste stap richting een meer geautomatiseerde en contextuele benadering van threat detection.


